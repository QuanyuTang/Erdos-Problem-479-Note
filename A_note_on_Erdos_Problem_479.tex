\documentclass[11pt,letterpaper,reqno]{amsart}
\usepackage{tikz}
\usetikzlibrary{positioning, shapes.geometric, arrows.meta, calc, positioning}
\usepackage{amssymb}
\usepackage{amsmath}
\usepackage{amsthm}
\usepackage{amsfonts}
\usepackage{mathtools}
\usepackage{bbm}
\usepackage{enumitem} 
\usepackage{pgfplots}
\pgfplotsset{compat=1.18} 
\usepackage{booktabs}
\usepackage{graphicx}
\usepackage[T1]{fontenc}
\usepackage{doi}
\usepackage{float} 
\addtolength{\hoffset}{-1.5cm}\addtolength{\textwidth}{3cm}
\addtolength{\voffset}{-1cm}\addtolength{\textheight}{2cm}

\usepackage{bookmark}
\usepackage{hyperref}
\hypersetup{pdfstartview={FitH}}
\newcommand{\C}{\mathbb{C}}
\newcommand{\cE}{\mathcal{E}}
\newcommand{\norm}[1]{\lVert #1 \rVert}
\newcommand{\abs}[1]{| #1 |}
\newcommand{\bv}{\mathbf{v}}
\newcommand{\bw}{\mathbf{w}}
\newcommand{\tr}{\operatorname{Tr}}
\DeclareMathOperator{\rank}{rank}

\newtheorem{thm}{Theorem}[section]
\newtheorem{lem}[thm]{Lemma}
\newtheorem{prop}[thm]{Proposition}
\newtheorem{cor}[thm]{Corollary}
\newtheorem{claim}{Claim}
\newtheorem{ques}[thm]{Question}
\newtheorem{prob}[thm]{Problem}
\newtheorem{conj}[thm]{Conjecture}
\theoremstyle{definition}
\newtheorem{exm}[thm]{Example}
\newtheorem{remark}[thm]{Remark}
\newtheorem{defn}[thm]{Definition}
\numberwithin{equation}{section}
\newcommand{\N}{\mathbb{N}}
\newcommand{\taufunc}{\tau}
\newcommand{\omegap}{\omega}
\newcommand{\ord}{\operatorname{ord}}
\newcommand{\R}{\mathbb{R}}        % real numbers
\newcommand{\E}{\mathbb{E}}        % expectation
\newcommand{\Var}{\mathrm{Var}}    % variance
\newcommand{\Cov}{\operatorname{Cov}}
\newcommand{\PP}{\mathbb{P}}       % probability
\newcommand{\eps}{\varepsilon}     % epsilon
\newcommand{\ind}{\mathbf{1}}      % indicator function
\newcommand{\seq}[1]{\left(#1\right)} % sequence
\makeatother

\begin{document}

\title{A Note on Erd\H{o}s Problem~\#479:\\
Infinitude of the Sets $A(2^i)$ and Related Results}


\author[Q.~Tang]{Quanyu Tang}
\date{\today}

\address{School of Mathematics and Statistics, Xi'an Jiaotong University, Xi'an 710049, P. R. China}
\email{tang\_quanyu@163.com}

% \subjclass[2020]{}

% \keywords{}

% \begin{abstract}

% \end{abstract}

\maketitle

\section{Introduction}

For an integer $k$ we consider the congruence
\begin{equation}\label{eq:main-congruence}
2^n \equiv k \pmod n,
\end{equation}
and the associated set
\[
A(k) \coloneqq \{ n \ge 1 : 2^n \equiv k \pmod n\}.
\]

In \cite[p.~96]{ErdosGraham}, Graham proposed the following conjecture, which also appears as Problem~\#479 on Bloom’s Erd\H{o}s Problems website~\cite{EP479}.
\begin{conj}\label{conj:479}
Is it true that, for every integer $k\neq 1$, the set $A(k)$ is infinite?
Equivalently, is it true that, for all $k\neq 1$, there are infinitely many $n$ such that $2^n\equiv k\pmod{n}$?
\end{conj}

It is easy to see that $2^n\not\equiv 1\pmod{n}$ for all $n>1$ (see, e.g., the proof reproduced on the OEIS wiki~\cite{OEIS2nmodn}), so the restriction $k\neq 1$ is necessary. In fact $A(1)=\{1\}$.

In their monograph~\cite[p.~96]{ErdosGraham}, Erd\H{o}s and Graham attribute the following partial result to Graham, Lehmer and Lehmer:
\begin{quote}
\emph{This is known to be true (see [Gr-Leh-Leh(xx)]) if $k=2^i$, $i\geq 1$, and $k=-1$.}
\end{quote}
However, [Gr-Leh-Leh(xx)] refers to an unpublished manuscript, and, as noted on Bloom’s Erd\H{o}s Problems website~\cite{EP479}, no published version of this manuscript seems to exist; it also does not appear in the standard bibliographies of Graham or D.~H.\ Lehmer.

We remark that the statement for $k=-1$ is now standard: numbers $n$ with $n\mid 2^n+1$ are called Nov\'ak numbers, and their infinitude is well established (see Kalmynin~\cite{Kalmynin} and OEIS~\cite{A006521}).

The main purpose of this note is twofold:
\begin{itemize}
  \item in Section~\ref{sec:2} we survey the existing results and record those integers $k$ for which $A(k)$ is currently known to be infinite;
  \item and, for completeness, in Section~\ref{sec:2^i} we give an explicit proof of the case $k=2^i$, $i\ge1$.
\end{itemize}



We do \emph{not} claim any of the underlying number-theoretic statements as new; the novelty here is only expository. The proof in Section~\ref{sec:2^i} is an independent argument using multiplicative orders and Dirichlet's theorem, and may or may not coincide with the (unpublished) proof of Graham--Lehmer--Lehmer~\cite{GrLehLeh}.



% For $k=-2$ the existence of infinitely many $n$ with $n\mid 2^n+2$ is a consequence of work of Li et al.\ on an IMO problem~\cite{LiExcalibur}, and is recorded as OEIS~\cite{A006517}.


\section{Known results and OEIS data}\label{sec:2}

\subsection{The sets \texorpdfstring{$A(k)$}{A(k)} and OEIS entries}

For each fixed integer $k$, the set $A(k)$ collects all positive integers $n$ with
\[
n \mid 2^n - k.
\]
The online database OEIS contains separate entries for many of these sets. A convenient starting point is the OEIS wiki page \emph{$2^n \bmod n$}~\cite{OEIS2nmodn} and the cross-references on the page for $A334634$~\cite{A334634}. In particular, the following table of integer values of $k$ for which $A(k)$ has its own OEIS entry is adapted from the table on the OEIS wiki page~\cite{OEIS2nmodn}.


\begin{table}[H]
\centering
\begin{tabular}{r l}
\toprule
$k$ & OEIS entry for $A(k)$ \\
\midrule
$-11$ & \href{https://oeis.org/A334634}{A334634} \\
$-10$ & \href{https://oeis.org/A245594}{A245594} \\
$-9$  & \href{https://oeis.org/A240942}{A240942} \\
$-8$  & \href{https://oeis.org/A245319}{A245319} \\
$-7$  & \href{https://oeis.org/A240941}{A240941} \\
$-6$  & \href{https://oeis.org/A245728}{A245728} \\
$-5$  & \href{https://oeis.org/A245318}{A245318} \\
$-4$  & \href{https://oeis.org/A244673}{A244673} \\
$-3$  & \href{https://oeis.org/A015940}{A015940} \\
$-2$  & \href{https://oeis.org/A006517}{A006517} \\
$-1$  & \href{https://oeis.org/A006521}{A006521} \\
$0$   & \href{https://oeis.org/A000079}{A000079} \\
\midrule
$2$   & \href{https://oeis.org/A015919}{A015919} \\
$3$   & \href{https://oeis.org/A050259}{A050259} \\
$4$   & \href{https://oeis.org/A015921}{A015921} \\
$5$   & \href{https://oeis.org/A128121}{A128121} \\
$6$   & \href{https://oeis.org/A128122}{A128122} \\
$7$   & \href{https://oeis.org/A033981}{A033981} \\
$8$   & \href{https://oeis.org/A015922}{A015922} \\
$9$   & \href{https://oeis.org/A051447}{A051447} \\
$10$  & \href{https://oeis.org/A128123}{A128123} \\
$11$  & \href{https://oeis.org/A033982}{A033982} \\
$12$  & \href{https://oeis.org/A128124}{A128124} \\
$13$  & \href{https://oeis.org/A051446}{A051446} \\
$14$  & \href{https://oeis.org/A128125}{A128125} \\
$15$  & \href{https://oeis.org/A033983}{A033983} \\
$16$  & \href{https://oeis.org/A015924}{A015924} \\
$17$  & \href{https://oeis.org/A124974}{A124974} \\
$18$  & \href{https://oeis.org/A128126}{A128126} \\
$19$  & \href{https://oeis.org/A125000}{A125000} \\
\midrule
$32=2^5$    & \href{https://oeis.org/A015925}{A015925} \\
$64=2^6$    & \href{https://oeis.org/A015926}{A015926} \\
$128=2^7$   & \href{https://oeis.org/A015927}{A015927} \\
$256=2^8$   & \href{https://oeis.org/A015929}{A015929} \\
$512=2^9$   & \href{https://oeis.org/A015931}{A015931} \\
$1024=2^{10}$ & \href{https://oeis.org/A015932}{A015932} \\
$2048=2^{11}$ & \href{https://oeis.org/A015935}{A015935} \\
$4096=2^{12}$ & \href{https://oeis.org/A015937}{A015937} \\
\bottomrule
\end{tabular}
\caption{Integer values of $k$ for which $A(k)$ currently has a dedicated OEIS entry (from the cross-references in~\cite{A334634}).}
\label{tab:OEIS}
\end{table}
For $k=1$ one has $A(1)=\{1\}$; this trivial case is discussed on the OEIS wiki~\cite{OEIS2nmodn}, but (reasonably) does not have its own numbered sequence.

\subsection{Values of \texorpdfstring{$k$}{k} with \texorpdfstring{$A(k)$}{A(k)} known to be infinite}

We now summarize what is currently known about the infinitude of $A(k)$.

\subsubsection{The trivial cases}

\begin{itemize}
  \item $k=0$: Here $2^n\equiv 0\pmod n$ if and only if $n$ is a power of $2$. Thus
  \[
  A(0)=\{2^m : m\ge0\},
  \]
  and $A(0)$ is clearly infinite. This is recorded as \href{https://oeis.org/A000079}{A000079} in the OEIS.

  \item $k=2$: For every odd prime $p$ we have $2^{p-1}\equiv 1\pmod p$ and hence
  \[
  2^p\equiv 2\pmod p.
  \]
  Thus all odd primes lie in $A(2)$, so $A(2)$ is infinite. This is sequence \href{https://oeis.org/A015919}{A015919}. 
\end{itemize}

\subsubsection{The Nov\'ak numbers: \texorpdfstring{$k=-1$}{k=-1}}

Numbers $n$ with $n\mid 2^n+1$ are called \emph{Nov\'ak numbers}. They form OEIS sequence \href{https://oeis.org/A006521}{A006521}. It is easy to see, using the lifting-the-exponent (LTE) lemma, that
\[
3^{m+1} \mid 2^{3^m}+1
\]
for every $m\ge0$, so in particular $3^m\in A(-1)$ for all $m$. Kalmynin~\cite{Kalmynin} gives quantitative lower bounds for the counting function of Nov\'ak numbers, confirming that there are infinitely many such $n$. Thus the set $A(-1)$ is infinite.

\subsubsection{Numbers with \texorpdfstring{$n\mid 2^n+2$}{n|2^n+2}: \texorpdfstring{$k=-2$}{k=-2}}

The set
\[
A(-2)=\{n\ge1 : n\mid 2^n+2\}
\]
is OEIS \href{https://oeis.org/A006517}{A006517}. Kin Y.~Li et al.\ showed in \cite{LiExcalibur} that $A(-2)$ is infinite.



\subsubsection{The powers of two: \texorpdfstring{$k=2^i$}{k=2^i}}

Erd\H{o}s and Graham state (without proof) that Graham, Lehmer and Lehmer showed $A(2^i)$ is infinite for every integer $i\ge1$~\cite[p.~96]{ErdosGraham}. We have not been able to locate a published proof of this statement, nor the original Graham--Lehmer--Lehmer manuscript. In Section~\ref{sec:2^i} below we give an explicit proof that $A(2^i)$ is infinite for every $i\ge1$, using multiplicative orders and Dirichlet's theorem on primes in arithmetic progressions.

For $i=1$ this reduces to the case $k=2$ discussed above; for $i\ge2$ this seems not to be written down in detail elsewhere. The sequences $A(2^i)$ for $i=1,2,3,\dots$ correspond to the OEIS entries
\[
\text{A015919, A015921, A015922, A015924, A015925, A015926, \dots}, 
\]
cf.\ Table~\ref{tab:OEIS}.

\subsubsection{Other values of \texorpdfstring{$k$}{k}}

For the remaining $k$ appearing in Table~\ref{tab:OEIS} (for example $k=3,5,-3,-4,\dots$), the current state of knowledge is essentially experimental:

\begin{itemize}
  \item for many $k$ at least one solution $n\in A(k)$ is known, sometimes astronomically large (e.g.\ $k=3$ has a known solution $n=4700063497$, see \href{https://oeis.org/A050259}{A050259});
  \item but for no such $k$ (besides those listed above) has it been proved that $A(k)$ is infinite, nor even that $A(k)$ is nonempty beyond a finite list of experimentally found $n$.
\end{itemize}

In particular, Conjecture~\ref{conj:479} remains open for every fixed $k$ other than
\[
k\in\{0,1,-1,-2, 2^i : i\ge1\}.
\]

\section{The case \texorpdfstring{$k=2^i$}{k = 2^i}}\label{sec:2^i}

% In this section we give an explicit argument showing that for every fixed integer $i\ge1$ there are infinitely many $n$ with
% \[
% 2^n\equiv 2^i \pmod n.
% \]
% This is the case $k=2^i$ of Graham's conjecture. As mentioned above, the result is not new: Erd\H{o}s and Graham already state it (with attribution to Graham--Lehmer--Lehmer) in~\cite{ErdosGraham}. The proof below is presented for completeness and uses only standard facts about multiplicative orders and Dirichlet's theorem on primes in arithmetic progressions (see, e.g., Davenport~\cite{Davenport}).

\begin{thm}\label{thm:2^i}
Let $i\ge 1$ be an integer. Then there exist infinitely many positive integers $n$ such that
\[
2^n \equiv 2^i \pmod n.
\]
Equivalently, $A(2^i)$ is infinite for every $i\ge1$.
\end{thm}

\begin{proof}
Fix $i\ge1$. We shall construct infinitely many integers $n$ of the form
\[
n = i p,
\]
where $p$ runs over an infinite set of primes depending on $i$.

Let $p$ be an odd prime with $p\nmid i$, and set $n=ip$. By Fermat's little theorem,
\[
2^{p-1}\equiv 1\pmod p,
\]
hence
\[
2^p = 2\cdot 2^{p-1}\equiv 2\pmod p.
\]
Therefore
\[
2^n = 2^{ip} = (2^p)^i \equiv 2^i \pmod p
\]
for every such prime $p$. Thus
\[
p \mid (2^n-2^i)
\]
holds automatically for all odd primes $p$, independently of any further conditions.

Now, we must ensure additionally that
\[
i \mid (2^n-2^i).
\]
Write a prime power factorization
\[
i = 2^s \prod_{j=1}^t q_j^{e_j},
\]
where $s\ge0$, $t\ge0$, and $q_1,\dots,q_t$ are distinct odd primes. We have
\[
2^n - 2^i = 2^i\bigl(2^{i(p-1)}-1\bigr),
\]
so it suffices to guarantee that each odd prime power $q_j^{e_j}$ divides $2^{i(p-1)}-1$ (the $2$-power part will be handled separately). Since $\gcd(2,q_j)=1$, $2$ is invertible modulo $q_j^{e_j}$, and the condition
\[
2^{i(p-1)}\equiv1\pmod{q_j^{e_j}}
\]
is equivalent to the statement that the multiplicative order
\[
d_j \coloneqq \ord_{q_j^{e_j}}(2)
\]
divides $i(p-1)$. Let
\[
g_j \coloneqq \gcd(d_j,i),\qquad m_j \coloneqq \frac{d_j}{g_j}.
\]
Then $d_j\mid i(p-1)$ is equivalent to $m_j\mid (p-1)$: indeed, writing $d_j=g_j m_j$ and $i=g_j v_j$ with $\gcd(m_j,v_j)=1$, we have
\[
d_j \mid i(p-1)\iff g_j m_j \mid g_j v_j (p-1)\iff m_j\mid(p-1).
\]Thus, for each odd prime power $q_j^{e_j}\parallel i$, the requirement that $q_j^{e_j}$ divides $2^{i(p-1)}-1$ is equivalent to the congruence
\[
p\equiv 1\pmod{m_j}.
\] Since $2^s\mid i$ and $2^s\mid 2^i$, automatically
\[
2^s \mid 2^i\mid 2^i\bigl(2^{i(p-1)}-1\bigr) = 2^n-2^i
\]
for every $p$. Thus the $2$-part of $i$ imposes no restriction on $p$.


Let
\[
L \coloneqq \operatorname{lcm}(m_1,\dots,m_t),
\]
with the convention that $L=1$ if $t=0$ (i.e., if $i$ has no odd prime factors). If $p$ is a prime such that
\[
p\equiv 1\pmod L
\quad\text{and}\quad
p\nmid i,
\]
then for every $j$ we have $p\equiv 1\pmod{m_j}$ and hence $q_j^{e_j}\mid 2^{i(p-1)}-1$, as required. As we have just seen, $2^s$ automatically divides $2^n-2^i$, and we have already observed that $p\mid 2^n-2^i$ always holds. Moreover, by construction $\gcd(i,p)=1$, so the prime factors of $i$ and $p$ are disjoint. Thus the divisibility
\[
i \mid (2^n-2^i)
\quad\text{and}\quad
p \mid (2^n-2^i)
\]
combine to give
\[
ip \mid (2^n-2^i).
\]
In other words, for any such $p$,
\[
2^{ip} \equiv 2^i \pmod{ip},
\]
so $n=ip$ lies in $A(2^i)$.


It remains to show that there are infinitely many primes $p$ with
\[
p\equiv 1\pmod L,\qquad p\nmid i.
\]
Since $L\ge1$ is fixed and $\gcd(1,L)=1$, Dirichlet's theorem on primes in arithmetic progressions implies that there are infinitely many primes $p\equiv 1\pmod L$. Excluding the finitely many primes dividing $i$ still leaves infinitely many such $p$. For each of these primes $p$, the integer $n=ip$ satisfies $2^n\equiv2^i\pmod n$, and the resulting values $n$ are clearly distinct and unbounded. Hence $A(2^i)$ is infinite, as claimed.
\end{proof}

To make the mechanism of the proof of Theorem~\ref{thm:2^i} more transparent, let us examine in detail what the construction produces when $i=3$.


\begin{exm}
Taking $i=3$, we have $i=3=2^0\cdot 3$, so the only odd prime factor is $q_1=3$ with $e_1=1$. The multiplicative order of $2$ modulo $3$ is
\[
d_1 = \ord_3(2) = 2.
\]
Thus
\[
g_1 = \gcd(d_1,3) = 1,\qquad m_1 = \frac{d_1}{g_1} = 2,
\]
and hence
\[
L = \operatorname{lcm}(m_1) = 2.
\]
According to the proof of Theorem~\ref{thm:2^i}, any prime $p$ with
\[
p \equiv 1 \pmod{L}, \qquad p \nmid i
\]
gives a solution $n = ip = 3p$. In this case $L=2$, so $p$ can be any odd prime different from $3$. For instance,
\[
p = 5,7,11,13,17,19,\dots
\]
yield
\[
n = 15,21,33,39,51,57,\dots,
\]
and one checks that $2^n \equiv 8 \pmod n$ for all these $n$.

Comparing with the OEIS entry \href{https://oeis.org/A015922}{A015922}, which begins
\[
1,2,3,4,8,9,15,21,33,39,51,57,63,69,\dots,
\]
we see that all the integers $3p$ produced by this construction indeed belong to $A(8)$.
\end{exm}





























\begin{thebibliography}{99}
\bibitem{EP479}
T. F. Bloom, Erd\H{o}s Problem~\#479, \url{https://www.erdosproblems.com/479}, accessed 2025-12-02



% \bibitem{Davenport}
% H.~Davenport, \textit{Multiplicative Number Theory}, 3rd ed., Graduate Texts in Mathematics, vol.~74, Springer, 2000.

\bibitem{ErdosGraham}
P.~Erd\H{o}s and R.~L.~Graham,
\textit{Old and New Problems and Results in Combinatorial Number Theory},
Monographie No.~28 de L'Enseignement Math\'ematique, Geneva, 1980.

\bibitem{GrLehLeh}
R.~L.~Graham, D.~H.~Lehmer and E.~Lehmer,
\emph{On values of $2^n$ modulo $n$}, unpublished manuscript; cited (as [Gr-Leh-Leh(xx)]) in~\cite{ErdosGraham}.


% \bibitem{Honsberger}
% R.~Honsberger, \textit{Mathematical Gems}, Dolciani Mathematical Expositions, MAA, 1973.

\bibitem{Kalmynin}
A.~Kalmynin, On Nov\'ak numbers,
\textit{Sb.\ Math.} \textbf{209} (2018), no.~4, 520--541; see also arXiv:\href{https://arxiv.org/abs/1611.00417}{1611.00417}.

\bibitem{LiExcalibur}
Kin Y.~Li et al.,
Solution to Problem 323,
\textit{Mathematical Excalibur} \textbf{14} (2009), no.~2, 3--4.
(\href{http://www.math.ust.hk/excalibur/v14_n2.pdf}{online})

% \bibitem{Sierpinski}
% W.~Sierpi\'nski,
% \textit{250 Problems in Elementary Number Theory},
% Elsevier, New York, 1970.

\bibitem{OEIS2nmodn}
OEIS Wiki, \textit{$2^n \bmod n$},
The On-Line Encyclopedia of Integer Sequences,
\href{https://oeis.org/wiki/2%5En_mod_n}{https://oeis.org/wiki/2\%5En\_mod\_n}.

\bibitem{A006521}
N.~J.~A.~Sloane et al.,
OEIS sequence \href{https://oeis.org/A006521}{A006521}:
Numbers $n$ such that $n$ divides $2^n+1$ (Nov\'ak numbers).

% \bibitem{A006517}
% N.~J.~A.~Sloane et al.,
% OEIS sequence \href{https://oeis.org/A006517}{A006517}:
% Numbers $n$ such that $n$ divides $2^n+2$.

% \bibitem{MOA006517}
% M.~Alekseyev,
% \emph{A006517: integers with $n\mid 2^n+2$},
% MathOverflow discussion and OEIS comments, 2012--2019.

\bibitem{A334634}
M.~Alekseyev,
OEIS sequence \href{https://oeis.org/A334634}{A334634}:
Numbers $m$ such that $m$ divides $2^m+11$, with detailed cross-references to all integer $k$ for which $A(k)$ has an OEIS entry.

\end{thebibliography}

\end{document}
